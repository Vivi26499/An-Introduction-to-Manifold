\documentclass[en, oneside]{vivi}

\ProjectInfos{An Introduction to Manifold}{MATH-512}{Spring, 2025}{Chapter 1}{Euclidean Spaces}{Vivi}[https://github.com/Vivi26499]{24S153073}
\begin{document}

\section{Smooth Functions on a Euclidean Space}
The calculus of $C^\infty$ functions will be our primary tool for studying higher-dimensional manifolds.

\mathsubsection{$C^{\infty}$ Analytic Functions}{C-infinity Analytic Functions}

Let $p = (p^1, \cdots, p^n)$ be a point in an open subset $U \subseteq \mathbb{R}^n$. 
\begin{dfn}
    Let $k$ be a non-negative integer. A real-valued function $f: U \to \mathbb{R}$ is said to be $C^k$ at $p \in U$ if its partial derivatives
    \begin{equation*}
        \frac{\partial^j f}{\partial x_1^{i_1} \cdots \partial x_n^{i_n}}
    \end{equation*}
    of all orders $j \leq k$ exist and are continuous at $p$.\\
    The function $f: U \to \mathbb{R}$ is $C^\infty$ at $p$ if it is $C^k$ at $p$ for all $k \geq 0$.\\
    A vector-valued function $f: U \to \mathbb{R}^m$ is said to be $C^k$ at $p$ if all of its component functions $f^1, \cdots, f^m$ are $C^k$ at $p$.\\
    $f: U \to \mathbb{R}$ is said to be $C^k$ on $U$ if it is $C^k$ at every point $p \in U$.\\
    The set of all $C^\infty$ functions on $U$ is denoted by $C^\infty(U)$ or $\mathcal{F}(U)$.
\end{dfn}
The function $f$ is real-analytic at $p$ if in some neighborhood of $p$, it is equal to its Taylor series at $p$.\\
A real-analytic function is necessarily $C^\infty$. However, the converse is not true. A $C^\infty$ function may fail to be real-analytic.

\subsection{Taylor's Theorem with Remainder}
\begin{dfn}
    A subset $S \subseteq \mathbb{R}^n$ is star-shaped with respect to a point $p \in S$ if for every point $x \in S$, the line segment from $p$ to $x$ lies in $S$. 
\end{dfn}
\begin{lem}
    Let $f$ be a $C^\infty$ function on an open subset $U \subseteq \mathbb{R}^n$ star-shaped with respect to a point $p = (p^1, \cdots, p^n) \in U$. 
    Then there are functions $g_1(x), \cdots, g_n(x) \in C^\infty(U)$ such that
    \begin{equation*}
        f(x) = f(p) + (x^i - p^i) g_i(x), \quad g_i(p) = \frac{\partial f}{\partial x^i}(p) 
    \end{equation*}
\end{lem}
If $f$ is a $C^\infty$ function on an open subset $U$ containing $p$, then there is an $\epsilon > 0$ such that 
\begin{equation*}
    p \in B(p, \epsilon) \subset U.
\end{equation*}
where $B(p, \epsilon) = \{ x \in \mathbb{R}^n : \left\lVert x-p \right\rVert < \epsilon \}$ is the open ball of radius $\epsilon$ centered at $p$.

\mathsection{Tangent Vectors in $\mathbb{R}^n$ as Derivations}{Tangent Vectors in Rn as Derivations}
In this section, we will find a characterization of tangent vectors in $\mathbb{R}^n$ that will generalize to manifolds.

\subsection{The Directinal Derivative}
To distinguish between points and vectors, we write a point in $\mathbb{R}^n$ as $p = (p^1, \cdots, p^n)$ and a vector in the tangent space $T_p\mathbb{R}^n$ as
\begin{equation*}
    v = \begin{bmatrix*}
        v^1 \\
        \vdots \\
        v^n
    \end{bmatrix*}
    \quad \text{or} \quad v = \left\langle
        \begin{matrix}
        v^1, \cdots, v^n
        \end{matrix}
        \right\rangle
\end{equation*}
We usually denote the standard basis for $\mathbb{R}^n$ or $T_p\mathbb{R}^n$ by $e_1, \cdots, e_n$, then $v = v^i e_i$ for some $v^i \in \mathbb{R}$.\\
The line through $p = (p^1, \cdots, p^n)$ in the direction of $v = (v^1, \cdots, v^n)$ in $\mathbb{R}^n$ has parametrization
\begin{equation*}
    c(t) = (p^1 + tv^1, \cdots, p^n + tv^n).
\end{equation*}
If $f$ is $C^\infty$ in a neighborhood of $p$ in $\mathbb{R}^n$ and $v \in T_p\mathbb{R}^n$, the \textbf{directional derivative} of $f$ at $p$ in the direction of $v$ is defined to be
\begin{align*}
    D_v f &= \lim_{t \to 0} \frac{f(c(t)) - f(c(0))}{t}\\
    &= \frac{\dif}{\dif t} \bigg|_{t=0} f(c(t))\\
    &= \frac{\dif c^i}{\dif t}(0) \frac{\partial f}{\partial x^i}(p) \quad \text{(by chain rule)}\\
    &= v^i \frac{\partial f}{\partial x^i}(p).
\end{align*}
We write
\begin{equation*}
    D_v = v^i \frac{\partial}{\partial x^i} \bigg|_{p}
\end{equation*}
for the map that sends a function $f$ to its directional derivative $D_v f$.\\
The association $v \mapsto D_v$ offers a way to characterize tangent vectors as certain oprerators on functions.
\subsection{Germs of Functions}
\begin{dfn}
    A \textbf{relation} on a set $S$ is a subset $R$ of $S \times S$. Given $x, y \in S$, we write $x \sim y$ if and only if $(x, y) \in R$.\\
    A relation $R$ is an \textbf{equivalence relation} if it satisfies the following properties for all $x, y, z \in S$:
    \begin{enumerate}[label=(\roman*)]
        \item \textbf{Reflexivity:} $x \sim x$,
        \item \textbf{Symmetry:} If $x \sim y$, then $y \sim x$,
        \item \textbf{Transitivity:} If $x \sim y$ and $y \sim z$, then $x \sim z$.
    \end{enumerate}
\end{dfn}
Consider the set of all pairs $(f, U)$ where $U$ is a neighborhood of $p$ and $f: U \to \mathbb{R}$ is a $C^\infty$ function. 
We say that $(f, U)$ is \textbf{equivalent} to $(g, V)$ if there exists a neighborhood $W \subseteq U \cap V$ such that $f|_W = g|_W$.
\begin{dfn}
    The \textbf{germ} of $f$ at $p$ is the equivalence class of the pair $(f, U)$.\\
    We write $C_p^\infty(\mathbb{R}^n)$, or simply $C^\infty_p$, for the set of all germs of $C^\infty$ functions on $\mathbb{R}^n$ at $p$.
\end{dfn}
\begin{dfn}
    An \textbf{algebra} over a field $K$ is a vector space $A$ over $K$ with a multiplication map
    \begin{equation*}
        \mu: A \times A \to A,
    \end{equation*}
    usually written $\mu(a, b) = a \cdot b$, that satisfies the following properties for all $a, b, c \in A$ and $r \in K$:
    \begin{enumerate}[label=(\roman*)]
        \item \textbf{Associativity:} $(a \cdot b) \cdot c = a \cdot (b \cdot c)$,
        \item \textbf{Distributivity:} $a \cdot (b + c) = a \cdot b + a \cdot c$ and $(a + b) \cdot c = a \cdot c + b \cdot c$,
        \item \textbf{Homogeneity:} $r(a \cdot b) = (ra) \cdot b = a \cdot (rb)$.
    \end{enumerate}
    Usually we write the multiplication as simply $a b$ instead of $a \cdot b$.
\end{dfn}
\begin{dfn}
    A map $L: V \to W$ between two vector spaces over the field $K$ is said to be a \textbf{linear map} or a \textbf{linear operator} if for all $u, v \in V$ and $r \in K$,
    \begin{enumerate}[label=(\roman*)]
        \item $L(u + v) = L(u) + L(v)$,
        \item $L(r u) = r L(u)$.
    \end{enumerate}
    To emphasize the scalars are in the field $K$, such a map is said to be \textbf{$K$-linear}.
\end{dfn}
\begin{dfn}
    If $A$ and $A'$ are algebras over a field $K$, a \textbf{algebra homomorphism} is a linear map $L: A \to A'$ that preserves the algebra multiplication: $L(ab) = L(a)L(b)$ for all $a, b \in A$.
\end{dfn}
The addition and multiplication of functions induce corresponding operations on $C_p^\infty$, making it into an algebra over $\mathbb{R}$.

\subsection{Derivations at a point}
For each tangent vector $v \in T_p\mathbb{R}^n$, the directional derivative at $p$ gives a map
\begin{equation*}
    D_v: C_p^\infty \to \mathbb{R}.
\end{equation*}
\begin{dfn}
    A linear map $D: C_p^\infty \to \mathbb{R}$ is called a \textbf{derivation} at $p$ or a \textbf{point derivation} if it satisfies the Leibniz rule:
    \begin{equation*}
        D(fg) = D(f)g(p) + f(p)D(g)
    \end{equation*}
    Denote the set of all derivations at $p$ by $\mathcal{D}_p(\mathbb{R}^n)$, which is a vector space over $\mathbb{R}$.
\end{dfn}
Obviously, the directional derivatives at $p$ are all derivations at $p$, so there is a map
\begin{align*}
    \phi: T_p(\mathbb{R}^n) &\to \mathcal{D}_p(\mathbb{R}^n),\\
    v &\mapsto D_v = v^i \frac{\partial}{\partial x^i} \bigg|_{p}.
\end{align*}
Since $D_v$ is clearly linear in $v$, $\phi$ is a linear map of vector spaces.
\begin{lem}
    If $D$ is a point-derivation of $C_p^\infty$, then $D(c) = 0$ for any constant function $c$.
\end{lem}
\begin{pf}
    By $\mathbb{R}$-linearity, $D(c) = c D(1)$. By the Leibniz rule, we have
    \begin{align*}
        D(1) &= D(1 \cdot 1)\\
        &= D(1) \cdot 1(p) + 1(p) \cdot D(1)\\
        &= 2 D(1),
    \end{align*}
    which implies that $D(1) = 0$, and therefore $D(c) = c D(1) = c \cdot 0 = 0$.\\
\end{pf}
\begin{lem}
    The map $\phi: T_p(\mathbb{R}^n) \to \mathcal{D}_p(\mathbb{R}^n)$ is an isomorphism of vector spaces.
\end{lem}
\begin{pf}
    To show that $\phi$ is injective, suppose $\phi(v) = D_v = 0$ for some $v \in T_p(\mathbb{R}^n)$. For the coordinate functions $x^j$, we have
    \begin{align*}
        0 = D_v x^j &= v^i \frac{\partial x^j}{\partial x^i} \bigg|_{p}\\
        &= v^i \delta^j_i\\
        &= v^j,
    \end{align*}
    which implies that $v = 0$. Thus, $\phi$ is injective.\\
    To show that $\phi$ is surjective, let $D \in \mathcal{D}_p(\mathbb{R}^n)$ and let $(f, V)$ be a representative of a germ in $C_p^\infty$.
    We may assume $V$ is an open ball, hence star-shaped. From Taylor's theorem with remainder, we have
    \begin{equation*}
        f(x) = f(p) + (x^i - p^i) g_i(x), \quad g_i(p) = \frac{\partial f}{\partial x^i}(p).
    \end{equation*}
    Applying $D$ to both sides, we get
    \begin{align*}
        D(f(x)) &= D[f(p)] + D[(x^i - p^i) g_i(x)]\\
        &= (Dx^i)g_i(p) + (p^i - p^i) Dg_i(x)\\
        &= (Dx^i)g_i(p)\\
        &= (Dx^i) \frac{\partial f}{\partial x^i}(p),
    \end{align*}
    which gives $D = D_v$ for $v = \langle Dx^1, \cdots, D x^n \rangle$. Thus, $\phi$ is surjective.
\end{pf}
Under this vector space isomorphism $T_p(\mathbb{R}^n) \simeq \mathcal{D}_p(\mathbb{R}^n)$, we can identify tangent vectors with derivations at $p$,
and the standard basis $e_1, \cdots, e_n$ of $T_p(\mathbb{R}^n)$ with the set $\frac{\partial}{\partial x^1}|_p, \cdots, \frac{\partial}{\partial x^n}|_p$ of partial derivatives,
\begin{align*}
    v &= \langle v^1, \cdots, v^n \rangle\\
    &= v^i e_i\\
    &= v^i \frac{\partial}{\partial x^i} \bigg|_{p}.
\end{align*}

\subsection{Vevtor Fields}
\begin{dfn}
    A \textbf{vector field} on an open subset $U \subseteq \mathbb{R}^n$ is a function that assigns to each point $p \in U$ a tangent vector $X_p \in T_p(\mathbb{R}^n)$.
    Since $T_p(\mathbb{R}^n)$ has basis $\frac{\partial}{\partial x^i}|_p$, we can write
    \begin{equation*}
        X_p = a^i(p) \frac{\partial}{\partial x^i} \bigg|_p, \quad a^i(p) \in \mathbb{R}.
    \end{equation*}
    Omitting $p$, we can write 
    \begin{equation*}
        X = a^i \frac{\partial}{\partial x^i} \quad \leftrightarrow \quad \begin{bmatrix*}
            a^1 \\
            \vdots \\
            a^n
        \end{bmatrix*},
    \end{equation*}
    where $a^i$ are functions on $U$. 
    We say that $X$ is $C^\infty$ on $U$ if all the coefficient functions $a^i$ are $C^\infty$ on $U$.\\
    The set of all $C^\infty$ vector fields on $U$ is denoted by $\mathfrak{X}(U)$.
\end{dfn}
\begin{dfn}
    If $R$ is a commutative ring with identity, a (left) $R$-module is an abelian group $A$ with a scalar multiplication
    \begin{equation*}
        \mu: R \times A \to A,
    \end{equation*}
    usually written $\mu(r, a) = ra$, such that for all $r, s \in R$ and $a, b \in A$,
    \begin{enumerate}[label=(\roman*)]
        \item \textbf{Associativity:} $(rs)a = r(sa)$,
        \item \textbf{Identity:} $1a = a$,
        \item \textbf{Distributivity:} $r(a + b) = ra + rb$ and $(r + s)a = ra + sa$.
    \end{enumerate}
\end{dfn}
$\mathfrak{X}(U)$ is a module over the ring $C^\infty(U)$ with the multiplication defined pointwise:
\begin{equation*}
    (fX)_p = f(p)X_p, \quad f \in C^\infty(U), \quad X \in \mathfrak{X}(U), \quad p \in U.
\end{equation*}
\begin{dfn}
    Let $A$ and $A'$ be $R$-modules. An $R$-\textbf{module homomorphism} from $A$ to $A'$ is a map $f: A \to A'$ that preserves 
    both the addition and the scalar multiplication: for all $a, b \in A$ and $r \in R$,
    \begin{enumerate}[label=(\roman*)]
        \item $f(a + b) = f(a) + f(b)$,
        \item $f(ra) = rf(a)$.
    \end{enumerate}
\end{dfn}

\subsection{Vector Fields as Derivations}
If $X \in \mathfrak{X}(U)$ and $f \in C^\infty(U)$, we can define a new function $Xf$ by
\begin{equation*}
    (Xf)(p) = X_pf \quad \text{for all } p \in U.
\end{equation*}
Writing $X = a^i \frac{\partial}{\partial x^i}$, we have
\begin{equation*}
    (Xf)(p) = a^i(p) \frac{\partial f}{\partial x^i}(p),
\end{equation*}
or
\begin{equation*}
    Xf = a^i \frac{\partial f}{\partial x^i},
\end{equation*}
which is a $C^\infty$ function on $U$. Thus, a $C^\infty$ vector field $X$ induces an $\mathbb{R}$-linear map
\begin{align*}
    X: C^\infty(U) &\to C^\infty(U),\\
    f &\mapsto Xf.
\end{align*}
$X(fg)$ satisfies the Leibniz rule:
\begin{equation*}
    X(fg) = (Xf)g + f(Xg).
\end{equation*}
\begin{dfn}
    If $A$ is an algebra over a field $K$, a \textbf{derivation} on $A$ is a $K$-linear map $D: A \to A$ that satisfies the Leibniz rule:
    \begin{equation*}
        D(ab) = (Da)b + a(Db) \quad \text{for all } a, b \in A.
    \end{equation*}
    The set of all derivations on $A$ is closed under addition and scalar multiplication and forms a vector space, denoted by Der$(A)$.
\end{dfn}
We therefore have a map
\begin{align*}
    \varphi: \mathfrak{X}(U) &\to \text{Der}(C^\infty(U)),\\
    X &\mapsto (f \mapsto Xf),
\end{align*}
which is an isomorphism of vector spaces, just as the map $\phi: T_p(\mathbb{R}^n) \to \mathcal{D}_p(\mathbb{R}^n)$.
\end{document}